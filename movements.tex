\achapter{Movements}{chap:movements}

A lot of this comes out of the Barry Harris idea of borrowing from the notes of the diminished scale that complements the chord you're on.  If you treat these notes as chord tones rather than brief passing tones, you can get some unusual sounding voicings (in a good way).

\section{Personal Favorites}

I'll often play a major ii-V like this:

\chords{
\chord{t}{x,x,p{3},p{5},p{4},p{2}}{{Dm11}}
\chord{t}{x,x,p{3},p{4},p{3},p{2}}{{G+7}}
\chord{t}{x,x,p{2},p{2},p{2},p{2}}{{C6/9}}
}

Or like this:

\chords{

}

Here's a nice way to end on a major 7th sound with some diminished borrowing:

\chords{
\chord{2}{x,p{3},p{2},p{4},p{4},x}{{Dmaj7}}
\chord{2}{x,p{3},p{2},p{1},p{4},x}{{D+}}
\chord{2}{x,p{3},p{2},p{2},p{4},x}{{D6}}
}

Moving the inner two voices of a diminished chord along the diminished scale.  Try running these voicings up and down with the low E string in the bass for a cool E7\ensuremath{\flat}9 sound that will resolve nicely to A major.

\chords{
\def\numfrets{7}
\chord{2}{n,p{3},p{3},p{1},p{3},x}{}
\chord{2}{n,p{3},p{4},p{2},p{3},x}{}
\chord{2}{n,p{3},p{6},p{4},p{3},x}{}
\chord{2}{n,p{3},p{7},p{5},p{3},x}{}
}

Here's a nice V-I resolution in D major, with the top voice moving chromatically down.
\chords{
\def\numfrets{7}
\chord{2}{p{3},x,p{3},p{4},p{4},p{3}}{{A13\ensuremath{\flat}9}}
\chord{2}{p{3},x,p{6},p{4},p{4},p{2}}{{A13\ensuremath{\flat}9}}
\chord{2}{x,p{3},p{2},p{2},p{2},p{1}}{{D6/9(\ensuremath{\sharp}11)}}
}

\section{Clare Fischer}

There's a reason Herbie Hancock cites this guy as an influence.

\subsection{Elizete with Cal Tjader}

Here's a minor ii-V in A minor.  

% The ear buys the Emaj7\ensuremath{\sharp}9 as a bitonal hybrid of E major (the V chord) and B7\ensuremath{\sharp}5 (the II-7 chord).

\chords{
\def\numfrets{6}
\chord{5}{p{2},p{3},p{2},p{2},p{3},x}{{Bm13\ensuremath{\flat}5}}
\chord{5}{n,p{6},p{4},p{3},p{2},x}{{Emaj7(\ensuremath{\sharp}9)}}
\chord{3}{n,p{2},p{3},p{2},p{2},x}{{E7(\ensuremath{\sharp}5)(\ensuremath{\flat}9)}}
\chord{3}{x,n,p{6},p{2},p{1},x}{{Am(add2)}}
}

And a major ii-V in C major (with an extra passing chord at the end):

\chords{
\def\numfrets{6}
\chord{3}{x,p{5},p{6},p{2},p{1},x}{{Dm13}}
\chord{3}{x,p{5},p{6},p{3},p{1},x}{{G13(\ensuremath{\sharp}11)}}
\chord{t}{x,p{3},p{2},p{4},p{2},p{4}}{{Cmaj13}}
\chord{t}{p{2},x,p{2},p{3},n,x}{{G\ensuremath{\flat}7(\ensuremath{\sharp}11)}}
}

\subsection{Morning}

I have a YouTube video lesson on these sequences.

\subsubsection{Verse}

In the first sequence, note the contrapuntal motion.  The soprano voice moves up chromatically while first the tenor voice and then the bass voice moves down chromatically.

\chords{
\def\numfrets{6}
\chord{6}{p{2},p{3},p{3},p{2},p{5},x}{{Cm7(\ensuremath{\flat}5)}}
\chord{6}{p{2},p{3},p{2},p{2},p{6},x}{{Cm7(\ensuremath{\flat}5)}}
\chord{6}{x,p{2},p{1},p{2},p{2},p{2}}{{F7(\ensuremath{\sharp}5)(\ensuremath{\sharp}9)}}
\chord{2}{x,x,p{5},p{4},p{3},p{2}}{{B\ensuremath{\flat}mi/maj7}}
\chord{2}{x,x,p{4},p{4},p{3},p{4}}{{B\ensuremath{\flat}m7}}
\chord{2}{x,x,p{3},p{4},p{3},p{5}}{{E\ensuremath{\flat}13}}
}

\chords{
\def\numfrets{6}
\chord{6}{p{2},p{3},p{3},p{2},p{5},x}{{Cm7(\ensuremath{\flat}5)}}
\chord{6}{p{2},p{3},p{2},p{2},p{6},x}{{Cm7(\ensuremath{\flat}5)}}
\chord{6}{x,p{2},p{1},p{2},p{2},p{2}}{{F7(\ensuremath{\sharp}5)(\ensuremath{\sharp}9)}}
\chord{2}{p{4},x,p{4},p{4},p{3},p{5}}{{B\ensuremath{\flat}m9}}
\chord{2}{x,p{4},p{3},p{4},p{3},p{2}}{{E\ensuremath{\flat}9(\ensuremath{\sharp}11)}}
}

\chords{
\chord{8}{p{3},x,p{3},p{3},p{2},x}{{E\ensuremath{\flat}m7}}
\chord{8}{x,p{3},p{2},p{2},p{1},x}{{A\ensuremath{\flat}13(\ensuremath{\flat}9)}}
\chord{5}{p{4},n,p{3},p{3},p{2}p{3},p{2}}{{D\ensuremath{\flat}maj13(\ensuremath{\sharp}11)}}
\chord{5}{x,p{2},p{3},p{3},p{1},p{2}}{{G\ensuremath{\flat}13(\ensuremath{\sharp}11)}}
}

\chords{
\chord{5}{p{3},x,p{3},p{4},p{3}p{4},p{5}}{{C13(\ensuremath{\sharp}9)}}
\chord{5}{x,p{3},p{2},p{3},p{1}p{3},p{3}}{{F7(\ensuremath{\sharp}5)(\ensuremath{\flat}9)(\ensuremath{\sharp}9)}}
}

\subsubsection{Bridge}

\subsection{Agogically So}

\subsection{Pensativa}