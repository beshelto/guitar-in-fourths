\achapter{Basic Grips}{chap:basic_grips}

This is a good place to start with chords.  I'll follow the convention from Joe Pass Guitar Chords and separate these into five general families of function (and sound).

\section{Root on 6th string, chord tones on strings 4-3-2}

Note that because of symmetry, all of these forms will work with the root on the 5th string and the chord tones on strings 3-2-1, albeit a fourth higher at a given fret position.

\subsection{Major}

\mediumchords
\def\numfrets{6}
\chords{
\chord{2}{p{3},x,p{5},p{4},p{2},x}{{A}}
\chord{2}{p{3},x,p{4},p{4},p{2},x}{{Amaj7}}
\chord{2}{p{3},x,p{4},p{4},p{4},x}{{Amaj13}}
\chord{2}{p{3},x,p{2},p{4},p{2},x}{{A6}}
\chord{2}{p{3},x,p{2},p{2},p{2},x}{{A6/9 (no 3rd)}}
}

\chords{
\chord{2}{p{3},x,p{4},p{2},p{2},x}{{Amaj9 (no 3rd)}}
\chord{2}{p{3},x,p{4},p{4},p{1},x}{{Amaj7(\ensuremath{\sharp}11)}}
\chord{2}{p{3},x,p{4},p{4},p{3},x}{{Amaj7(\ensuremath{\sharp}5)}}
}

\subsection{Minor}

\chords{
\chord{2}{p{3},x,p{5},p{3},p{2},x}{{Am}}
\chord{2}{p{3},x,p{3},p{3},p{2},x}{{Am7}}
\chord{2}{p{3},x,p{4},p{3},p{2},x}{{Am(maj7)}}
\chord{2}{p{3},x,p{5},p{3},p{4},x}{{Am6}}
\chord{2}{p{3},x,p{2},p{3},p{2},x}{{Am6}}
}

\chords{
\chord{2}{p{3},x,p{3},p{3},p{4},x}{{Am13}}
\chord{t}{p{5},x,p{5},p{5},p{2},x}{{Am11}}
}

\subsection{Dominant}
\chords{

}

\subsection{Diminished}

\subsection{Half Diminished}

\section{Root on 5th string, chord tones on strings 4-3-2}

\subsection{Major}

\subsection{Minor}

\subsection{Dominant}

\subsection{Diminished}

\subsection{Half Diminished}
